\documentclass[times, 10pt,twocolumn]{article} 
\usepackage{latex8}
\usepackage[utf8]{inputenc}
\usepackage{times}
\usepackage{graphicx}
\usepackage[font=small,labelfont=bf]{caption}
\pagestyle{empty}

%------------------------------------------------------------------------- 
\begin{document}


\title{HillClimb@Cloud \\ Cloud Computing and Virtualization \\MEIC-IST }

\author{83531 -Miguel Belém - miguelbelem@tecnico.ulisboa.pt\\
83567 - Tiago Gonçalves - tiago.miguel.c.g@tecnico.ulisboa.pt\\
83576 - Vítor Nunes - vitor.sobrinho.nunes@tecnico.ulisboa.pt
}

\maketitle
\thispagestyle{empty}

\begin{abstract}
   This report explains the second phase of the Cloud Computing and Virtualization
   course project. This project makes use of code instrumentation to obtain metrics 
   from a running worker so we can estimate the cost of a request on a Load Balancer 
   and Auto-Scaler.
\end{abstract}

\Section{Introduction}
   This project consists on developing a Load Balancer and Auto-Scaler for 
   a program running on Amazon Web Services that searches for the maximum value 
   of a map using different search strategies (A*, BFS, DFS). For that, we instrumented
   the code so that we could produce metrics about the running instances and adapt the system 
   to get the maximum value of the instances running in AWS at a minimum cost.

   Although this project uses a specific program to load the instances its objective 
   is to be able to generalize the procedure in a way that the Load balancer and 
   Auto-Scaler would work even if the given program is completely different given
   that the code running on the instances would be correctly instrumented.

%-------------------------------------------------------------------------

\Section{Architecture}
   We have two types of Instances. The Worker Instance, which runs the Web Server 
   that solves the requests made by the users to the Load Balancer. The Load Balancer 
   (Auto-Scaler and Metric Storage System) instance, which has the Load Balancer
   that distributes the requests from the clients to the running Worker Instances. 
   It also implements the Auto Scaling functions, such as launching (scale-up) and 
   terminating instances (scale-down). This instance is also runs the Web Server 
   for the MetricStorageServer which is the point of communication with our 
   table on DynamoDB.

\SubSection{Types of Instances}
%-------------------------------------------------------------------------
\subsubsection{Worker Instance}
   We created a Linux Image based on the Linux AMI AWS - t2.micro image, 
   which is the one eligible for free tier usage on AWS. It has a single core
   (up to 3.3 Ghz) and 1 GB of RAM. The Linux distribution was updated and
   extra packages needed for running the Web Server and BIT were installed.
   We load the Web Server that takes accepts requests on port 8000 and 
   altered on-boot scripts to start up the Web Server every time that
   the system is booted.

   We created an AMI so that we could replicate the instance and use it 
   on a Load Balancer. 

%-------------------------------------------------------------------------
\subsubsection{Manager Instance} 
   The Manager Image has the same properties as the Workers' Image, but 
   instead of running the Worker Web Server on boot, it executes the Load Balancer
   (from now on called LB) Web Server and Metric Storage Server (from now on 
   called MSS). The LB receives its requests on port 8001 and the MSS on the
   port 8002. All communication between Web Servers is made using HTTP requests
   using different contexts to execute distinct operations.
   
%-------------------------------------------------------------------------
\SubSection{Worker}
   The Worker's Web Server has 3 contexts to receive requests:
   \begin{itemize}
      \item /climb - receives the parameters for the search algorithm and runs it. 
      \item /ping - answers to a ping request 
      \item /progress - returns the progress of all the requests running on the machine
   \end{itemize}
   
\SubSection{Manager}
\subsubsection{Load Balancer}
\subsubsection{Auto Scaler}
   We created an Auto Scaling Group on AWS that increases the number of instances by
   1 if the average CPU utilization is over 60 \% in a period of 5 minutes and
   decreases the number of instances by 1 (to a minimum of 1 instance) if the average CPU 
   utilization of all instances is under 40\% for a period of 5 minutes. This allows
   to scale up in a situation of continuous heavy load while not scaling down instantly
   if we get a slight pause of incoming requests. 

   We want to be careful shuting down instances as they take some time starting up and if 
   we get a temporary decrease and terminate them and then we might get back to a state
   where we can't deal with all requests that turns to a higher response time to those 
   requests. To avoid this we wait to see if the decrease of requests (lower CPU usage)
   is not temporary (hence the 5 minutes under 40\% CPU load). Also if for some reason
   an instance becomes unhealthy (unresponsive or the Web Server crashed for some reason)
   then the auto-scaler will terminate that instance and start another.
\subsubsection{Metric Storage Manager}

%------------------------------------------------------------------------- 
\Section{Instrumentation}
   We used BIT (Bytecode Instrumenting Tool) presented in the laboratories to 
   to alter the Java Class Byte Code of the program so we could measure the metrics
   that we wanted in a way that would help us decide the cost of replying to a certain request.

   This metrics need to be heavily weighted as they might give a big overhead to the 
   the original program. To store this metrics we created a class called Metrics, which
   stores all of the counter for each type of metric. We guarantee that each thread only 
   counts its own metrics by using a ConcurrentHashMap to store all the metrics, where the 
   key is the Thread ID and the Value is the Metrics Object (containing all the metrics). At 
   the end of each request we store permanently the results of the instrumentation and remove
   the Metrics file from the HashMap, this allows to reset the counters for that specific thread.

   With the metrics it's also stored the parameters given to program so we can associate those with
   the load it creates and in the future compare them with new requests.


%-------------------------------------------------------------------------    
\SubSection{Metrics Used}
   We decided to try a few simple metrics. Instructions run, Basic Blocks, 
   Methods Called, Branches Queries, Branches Taken and Branches not taken. 
   We ran the instrumented Web Server on an instance of AWS and made some requests 
   (one at a time) while tracking the time it took to reply to each request.
   We obtained a table with all the values and concluded that some metrics grew
   linearly with the time to reply a request. Those metrics were Instructions
   Run, Basic Blocks, Branches and Branch not Taken.

   Methods Called and Branch Taken were inconsistent because they depended on 
   the search algorithm being used and do not necessarily mean a higher cost to 
   reply to a request as different methods could have different number of instructions
   which could mislead the cost of the request.

   As the other metrics scaled linearly with the time to reply to the request we can
   associate a higher number on this metrics to a more CPU intensive task.
   We decided to stick with Basic Blocks and Branches not Taken due to its 
   linear grow matched with response time to the request and also because they provided
   the lower overhead to the running code. 

%-------------------------------------------------------------------------
\SubSection{Storing the metrics}
\subsubsection{Real Cost Calculation}
\subsubsection{Cache}
   We noticed some latency when performing multiple requests to dynamoDB, in order to
   avoid some of this latency, a local cache was created in MSS. This cache as N entries (for 
   testing proposes is set for N=20, in a real system the value must be reviewed), it implements 
   last recently used politic, so when request 21 arrives, request 1 is deleted from cache. With this cache
   if a request arrives to MSS and it is in cache, we do not need to consult dynamoDB and the cost is instantly returned
   to the Load Balancer. 

\subsubsection{Cost Estimation}
   When a new request arrives and we already have some information in dynamoDB, we will try to estimate the cost of the 
   new one, based on the ones previously stored. To this we fetch from dynamo the closest ones. The following filters 
   are applied to classify if a request in dynamo is close:
	\begin{itemize}
	\item Same search algorithm;
	\item Same image;
	\item The initial point must be the same with a margin of + or - 10;
	\item search area must be the same with a margin of + or - 2000.
   With all the requests matched we perform an average of their costs and now that is the estimated cost of the new 
   request.
\end{itemize}

\subsubsection{Retrieving Cost}


\end{document}

